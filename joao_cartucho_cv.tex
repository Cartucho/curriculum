\documentclass{article}

\usepackage[portuguese,english]{babel}
\usepackage{eurosym}
\usepackage{iflang}
\usepackage{resume}
\usepackage{hyperref}
\usepackage{textcomp}

\begin{document}

\selectlanguage{english}

\name{\bf JO\~{A}O CARTUCHO}

\begin{llist}

% Personal Information
%\sectiontitle{Personal Information}
%\textit{\textnumero 7 R/C Dto, Rua Filinto El\'{i}sio, 1300-241, Lisbon, Portugal}
%\textit{Mobile:} 00351 913 113 753\\
%\textit{E-mail address: }\href{mailto:to.cartucho@gmail.com}{to.cartucho@gmail.com}\\

% Education
\sectiontitle{Education}

\employer{CARNEGIE MELLON UNIVERSITY} \location{Pittsburgh, PA, USA}
\dates{10/2017--12/2017}
\IfLanguageName{english}
{
Visiting Scholar, Machine Learning, Professor: Manuela Veloso
\begin{itemize}
\item[\textendash] Introduced an algorithm in which robots know how to address their visual perception limitations: they rely on proactively asking for help from humans in the environments for labeling the surrounding objects.
\end{itemize}
}
{
Erasmus.
}

\employer{POLITECNICO DI TORINO} \location{Torino, Italy}
\dates{09/2015--08/2016}
\IfLanguageName{english}
{
Erasmus Student, Computer Science, GPA: 26.44/30.00
\begin{itemize}
\item[\textendash] Able to develop complex data structures and ADTs (linked lists, queues, stacks, heaps, trees, hash tables and graphs) and related algorithms
\item[\textendash] Able to evaluate algorithm complexity and improve efficiency in terms of execution time and/or memory use
\item[\textendash] Able to solve optimization and network flow problems, using exact methods, heuristics or linear programming
\item[\textendash] Able to develop strategies for problem solving with deep learning, pattern recognition or neural networks
\item[\textendash] Able to develop computer vision applications
\end{itemize}
}
{
Erasmus.
}

\employer{INSTITUTO SUPERIOR T\'{E}CNICO, LISBON} \location{Lisbon, Portugal}
\dates{09/2013--now}
\IfLanguageName{english}
{
M.S. in Aero-Space Engineering.\\
Expected graduation date: 12, 2017\\
Current Master's Average (0-20): 17.44

\begin{itemize}
\item[\textendash] Able to fully develop all stages of the life-cycle of airplanes, helicopters, robotized aircrafts, spaceships and satellites, from its conception and project, to the operation and maintenance, tests and production.
\end{itemize}
}
{
(planeado) Mestrado Integrado em Engenharia Aeroespacial.\\
Data de gradua\c{c}\~{a}o esperada: 12, 2017
}

\employer{UNIVERSITY ENTRANCE EXAMINATION} \location{PORTUGAL}
National Exams (final grade 0-20):
\begin{itemize}
\item[\textendash] Mathematics A: 18.1
\item[\textendash] Physics and Chemistry: 19.8
\end{itemize}

% Internship Experience
\IfLanguageName{english}
{
\sectiontitle{Experience}
}
{
\sectiontitle{Exp\'{e}rience Professionnelle}
}
\vspace{-0.33cm}

\employer{GOOGLE SUMMER OF CODE 2017}\location{Lisbon, Portugal}
\dates{05/2017--08/2017}
\IfLanguageName{english}
{
\href{https://summerofcode.withgoogle.com/projects/\#6351260335734784}{OpenCV Documentation Improvement Tools}
\vspace{-0.33cm}
\begin{itemize}
\item[\textendash] HighGui module in Java;
\item[\textendash] Python code Tests (using Pylint);
\item[\textendash] Java compiling Tests (using Apache Ant);
\item[\textendash] Function/Constant/Class signatures in Python and Java;
\item[\textendash] 15 updated tutorials (added Java and Python codes).
\end{itemize}
}
{
OpenCV Documentation Improvement tools
\vspace{-0.33cm}
\begin{itemize}
 \item developing open source code.
\end{itemize}
}

\employer{GOOGLE SUMMER OF CODE 2016}\location{Torino, Italy}
\dates{05/2016--08/2016}
\IfLanguageName{english}
{
\href{https://summerofcode.withgoogle.com/archive/2016/projects/6414610965987328/}{Multi-language OpenCV Tutorials in Python, C++ and Java}
\vspace{-0.33cm}
\begin{itemize}
\item[\textendash] Choose Your Language Buttons;
\item[\textendash] Tools to Enhance Usability;
\item[\textendash] Java and Python translation of the code in C++.
\end{itemize}
}
{
Multi-language OpenCV Tutorials in Python, C++ and Java
\vspace{-0.33cm}
\begin{itemize}
 \item developing open source code.
\end{itemize}
}

% Project Activities
\IfLanguageName{english}
{
\sectiontitle{Project Activities}
\vspace{-0.4cm}

\employer{Drone Localization using EKF and CV} \location{Lisbon, Portugal}
\dates{09/2016--01/2017}
Course: Autonomous Systems.\\
Real-time pose estimation (position + orientation) of a Drone merely with its camera. This task was achieved by using an Extended Kalman Filter (EKF) and Computer Vision with data from aerial images.\\
Final grade 0-20 : 19 (best student in 2016/2017)

\employer{Fingerprint Badging System} \location{Torino, Italy}
\dates{02/2016--07/2016}
Course: Project and Laboratory on Communication	Systems.\\
Developing a Badging System Device using .NET Gadgeteer FEZ Spider Kit from GHI Electronics.\\
Final grade 0-30 : 30L (excellent)

\employer{Handwriting Digit Recognition} \location{Torino, Italy}
\dates{09/2015--07/2016}
Course: Computer Vision.\\
Developing Software to recognize handwritten digits in a photo.\\
Final Project grade 0-5 : 5 (best)

}
{
\sectiontitle{Activit\'{e}s Professionnelles}

Em Portugues

}

% Skills
\IfLanguageName{english}
{
\sectiontitle{Skills}
Programming Languages: C (grade 19/20), C++, C\# (grade 30/30), Java (grade 30/30), Matlab, Python, HTML, Javascript, jQuery\\
Extra Interests: OpenCV, ROS, Solid Works\\

Languages: Portuguese (native), English (fluent, Cambridge ESOL - FCE), Italian (fluent)\\
Volunteering in: Finland, Greece, Italy, Netherlands, Portugal, Romania, Turkey
}
{
\sectiontitle{Comp\'{e}tences}
{\em Comp\'{e}tences}: Vision, Robotique, Programmation, Architecte Logiciel, Management\\
{\em Langages de Programmation}: C++, Python\\
{\em Biblioth\`{e}ques Logicielles}: OpenCV, ROS, Boost, OpenMP, TBB, PVM, MPI \\
{\em Autres Int\'{e}r\^{e}ts}: Android, Drupal, Matlab, Javascript, PHP, NOSQL \\
{\em Langues}: Fran\c{c}ais (natif), Anglais (bilingue), Espagnol (fluent), Portugais (d\'{e}butant), Italien
(d\'{e}butant)
}

% Honors & Awards
\sectiontitle{Honors/Awards}
\textendash Academic Merit Diploma, IST - Aerospace, 2015/2016\\
\textendash FCT Scholarship, CMU\\
\textendash 2nd place, CMU's Annual Robotics Hackathon

% Publications
\sectiontitle{Publications}
\vspace{-0.3cm}
\employer{Towards a Robust Interactive and Learning Social Robot}\\
{\em Submitted to AAMAS 2018.}\\
Michiel de Jong, Kevin Zhang, Travers Rhodes, Aaron Roth, Robin Schmucker,
Chenghui Zhou, Sofia Ferreira, Jo\~{a}o Cartucho, Manuela Veloso.

\sectiontitle{References}

\textbf{Manuela Veloso}\\
Head, Machine Learning Department\\
Carnegie Mellon University\\
\href{mailto:mmv@cs.cmu.edu}{mmv@cs.cmu.edu}

\textbf{Vincent Rabaud}\\
Google Inc., Senior Software Engineer\\
OpenCV Foundation, Co-founder\\
\href{mailto:vincent.rabaud@gmail.com}{vincent.rabaud@gmail.com}

%PhD. Politecnico di Torino\\
%Dipartamento di Automatica e Informatica\\
%\href{mailto:valentina.gatteschi@polito.it}{valentina.gatteschi@polito.it}



\end{llist}

{\em Last update: \today}

\end{document}
