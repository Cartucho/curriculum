\documentclass{article}

\usepackage[portuguese,english]{babel}
\usepackage{eurosym}
\usepackage{iflang}
\usepackage{resume}
\usepackage{hyperref}
\usepackage{textcomp}

\begin{document}

\selectlanguage{english}

\name{\bf JO\~{A}O CARTUCHO}

\begin{llist}

% Personal Information
\sectiontitle{Personal Information}

\textit{\textnumero 7 R/C Dto, Rua Filinto El\'{i}sio, 1300-241, Lisbon, Portugal}

\textit{Mobile:} 00351 913 113 753\\
\textit{E-mail address: }\href{mailto:to.cartucho@gmail.com}{to.cartucho@gmail.com}\\

% Education
\sectiontitle{Education}

\employer{POLITECNICO DI TORINO} \location{Torino, TO, ITALY}
\dates{09/2015--08/2016}
\IfLanguageName{english}
{
Abroad as an Erasmus student in Computer Science Courses.
\begin{itemize}
\item[\textendash] Able to develop complex data structures and ADTs (linked lists, queues, stacks, heaps, trees, hash tables and graphs) and related algorithms
\item[\textendash] Able to evaluate algorithm complexity and improve efficiency in terms of execution time and/or memory use
\item[\textendash] Able to solve optimization and network flow problems, using exact methods, heuristics or linear programming
\item[\textendash] Able to develop strategies for problem solving with deep learning, pattern recognition or neural networks
\item[\textendash] Able to develop computer vision applications
\end{itemize}
}
{
Erasmus.
}

\employer{INSTITUTO SUPERIOR T\'{E}CNICO, LISBON} \location{Lisbon, LIS, PORTUGAL}
\dates{09/2013--now}
\IfLanguageName{english}
{
(planned) M.S. in Aero-Space Engineering.\\
Expected graduation date: 12, 2017\\
Current Master's Average (0-20): 17.44

\begin{itemize}
\item[\textendash] Able to fully develop all stages of the life-cycle of airplanes, helicopters, robotized aircrafts, spaceships and satellites, from its conception and project, to the operation and maintenance, tests and production.
\end{itemize}
}
{
(planeado) Mestrado Integrado em Engenharia Aeroespacial.\\
Data de gradua\c{c}\~{a}o esperada: 12, 2017
}

\employer{UNIVERSITY ENTRANCE EXAMINATION} \location{PORTUGAL}
National Exams (final grade 0-20):
\begin{itemize}
\item[\textendash] Mathematics A: 18.1
\item[\textendash] Physics and Chemistry: 19.8
\end{itemize}

% Internship Experience
\IfLanguageName{english}
{
\sectiontitle{Internship Experience}
}
{
\sectiontitle{Exp\'{e}rience Professionnelle}
}
\vspace{-0.33cm}

\employer{GOOGLE SUMMER OF CODE 2016}\location{Torino, Italy}
\dates{05/2016--08/2016}
\IfLanguageName{english}
{
\href{https://summerofcode.withgoogle.com/archive/2016/projects/6414610965987328/}{Multi-language OpenCV Tutorials in Python, C++ and Java}
\vspace{-0.33cm}
\begin{itemize}
 \item[\textendash] As a student, I worked as an open source code developer for this Google project. My project consisted of developing tools to improve the documentation and official tutorials of OpenCV, an open source computer vision library. Since OpenCV has had 10 million downloads so far the tutorials are seen by thousands of people every single day, which demonstrates the impact of this project.
\end{itemize}
}
{
Multi-language OpenCV Tutorials in Python, C++ and Java
\vspace{-0.33cm}
\begin{itemize}
 \item developing open source code.
\end{itemize}
}

% Project Activities
\IfLanguageName{english}
{
\sectiontitle{Project Activities}
\vspace{-0.4cm}

\employer{Fingerprint Badging System} \location{Torino, Italy}
\dates{02/2016--07/2016}
Course: Project and Laboratory on Communication	Systems.\\
Developing a Badging System Device using .NET Gadgeteer FEZ Spider Kit from GHI Electronics.\\
Final grade 0-30 : 30L (excellent)

\employer{Handwriting Digit Recognition} \location{Torino, Italy}
\dates{09/2015--07/2016}
Course: Computer Vision.\\
Developing Software to recognize handwritten digits in a photo.\\
Final Project grade 0-5 : 5

\employer{Clustering Algorithm of Vector Quantization} \location{Torino, Italy}
\dates{09/2015--07/2016}
Course: Artificial intelligence.\\
Clustering the MNIST 10-digit set database, obtaining a final error of 4.8\%\\
using a 10 minutes computation over the 60K-digits set.\\
Final grade 0-30 : 27

}
{
\sectiontitle{Activit\'{e}s Professionnelles}

Em Portugues

}

% Skills
\IfLanguageName{english}
{
\sectiontitle{Skills}
Programming Languages: C (grade 19/20), C++, C\# (grade 30/30), Java (grade 30/30), Matlab, Python, HTML, Javascript, jQuery\\
Extra Interests: OpenCV, CPLEX, Solid Works, Photoshop, Sony Vegas, Micro Framework.NET, ROS, Adobe Illustrator, Unity\\

Languages: Portuguese (native), English (fluent) - Cambridge ESOL - FCE, Italian (fluent), Spanish (beginner)\\
Volunteering and Interchanges: Finland, Greece, Italy, Netherlands, Portugal, Romania, Turkey
}
{
\sectiontitle{Comp\'{e}tences}
{\em Comp\'{e}tences}: Vision, Robotique, Programmation, Architecte Logiciel, Management\\
{\em Langages de Programmation}: C++, Python\\
{\em Biblioth\`{e}ques Logicielles}: OpenCV, ROS, Boost, OpenMP, TBB, PVM, MPI \\
{\em Autres Int\'{e}r\^{e}ts}: Android, Drupal, Matlab, Javascript, PHP, NOSQL \\
{\em Langues}: Fran\c{c}ais (natif), Anglais (bilingue), Espagnol (fluent), Portugais (d\'{e}butant), Italien
(d\'{e}butant)
}

\sectiontitle{References}

\textbf{Manuela Veloso}\\
Head, Machine Learning Department\\
Carnegie Mellon University\\
\href{mailto:mmv@cs.cmu.edu}{mmv@cs.cmu.edu}

\textbf{Vincent Rabaud}\\
Google Inc., Senior Software Engineer\\
OpenCV Foundation, Co-founder\\
\href{mailto:vincent.rabaud@gmail.com}{vincent.rabaud@gmail.com}

%PhD. Politecnico di Torino\\
%Dipartamento di Automatica e Informatica\\
%\href{mailto:valentina.gatteschi@polito.it}{valentina.gatteschi@polito.it}



\end{llist}

{\em Last update: \today}

\end{document}
